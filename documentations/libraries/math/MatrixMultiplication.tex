
% dont forget to latexmk -pdf -c MatrixMultiplication.tex

\documentclass{article} 
%\usepackage{graphicx} 
\usepackage{fancyhdr} 
\usepackage{lastpage} 
\usepackage{amsmath} 
\usepackage{amssymb} 
\usepackage[headsep=3em]{geometry} 
%\usepackage{tabularx} 
\usepackage{array} 
\usepackage{comment} 


\pagestyle{fancy} 
\fancyhf{} 

\fancyhead[L]{Matrix Multiplication Basics and Neural Networks}
\fancyhead[R]{\thepage of \pageref{LastPage} \\ Carson Crowley} 

\renewcommand{\thesection}{\Roman{section}} 


\begin{document} 

\begin{center}
	\text{} 
	\\[22em] 
	{\LARGE \textbf{Matrix Multiplication Basics and Neural Networks}} \\[1em] 
	{\large \textbf{March, 2025}} \\[0.5em] 
	{\small \textbf{Carson Crowley}} 
\end{center} 

\thispagestyle{empty} 
\newpage
\section{Summary} 
\textbf{Matrix Multiplication} is a \textbf{binary operation} in which the product of two \(m\cdot n\) matricies (\(A\times B\)) results in a third matrix \(C\). The resultant matrix will have a row size equal to the row size in matrix \(A\), and a column size equal to that of the column size of matrix \(B\).     
\subsection{Definition of Matrix Multiplication}
For an \(m\cdot n\) matrix \(A\), and an \(n\cdot p\) matrix \(B\): 
\begin{equation}
	A = \begin{pmatrix} a_{11} & a_{12} & \hdots & a_{1n} \\ 
			    a_{21} & a_{22} & \hdots & a_{2n} \\
			    \vdots & \vdots & \ddots & \vdots \\ 
			    a_{m1} & a_{m2} & \hdots & a_{mn} 
	\end{pmatrix},\text{ } 
	B = \begin{pmatrix} b_{11} & b_{12} & \hdots & b_{1p} \\
	 		    b_{21} & b_{22} & \hdots & b_{2p} \\ 
			    \vdots & \vdots & \ddots & \vdots \\ 
			    b_{n1} & b_{n2} & \hdots & b_{np} 
	\end{pmatrix}
\end{equation} 
there is a resultant matrix \textbf{\textit{C}} such that \(C\) = \textbf{\textit{AB}}: 
\begin{equation}
	C = \begin{pmatrix} c_{11} & c_{12} & \hdots & c_{1p} \\
			    c_{21} & c_{22} & \hdots & c_{2p} \\ 
			    \vdots & \vdots & \ddots & \vdots \\ 
		  	    c_{m1} & c_{m2} & \hdots & c_{mp} 
	\end{pmatrix} 
\end{equation} 
where
\[C_{ij} = a_{i1}b_{j1} + a_{i2}b_{j2} + \hdots + a_{in}b_{jn} = \sum_{k=1}^{n}a_{ik}b_{kj}\]
for \(i = 1 \hdots m\) and \(j = 1 \hdots p\). \\[0.5em] 
Importantly, \(c_{ij}\) is the \textbf{\textit{dot product}} of the \(i^{\text{th}}\) row of \textbf{\textit{A}} and the \(j^{\text{th}}\) column \textbf{\textit{B}}. 
		
		% TODO: input the rewritten C matrix dot product of AB 


\section{Computational Complexity} 

\end{document}  
